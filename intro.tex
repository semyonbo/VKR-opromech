%% Начало содержательной части.

%%Костыль, копировать этот блок для создания новой ГЛАВЫ, она же РАЗДЕЛ (для обычных подразделов - пункты и подпункты - можно не копировать)
% \begin{nohyphens}
% {
% \titleformat{\chapter}[block]{\normalsize\bfseries}{\thechapter~}%
% {0pt}{}[\thispagestyle{\TheOnlyTruePageStyle}]
% \titlespacing{\chapter}{1.25cm}{0pt}{0pt}

\chapter{Введение}

% \titleformat{\chapter}[block]{\normalsize\bfseries\center}{\thechapter~}%
% {0pt}{}[\thispagestyle{\TheOnlyTruePageStyle}]
% \titlespacing{\chapter}{0pt}{-30pt}{0pt}
% }
% \end{nohyphens}

%% Так помечается начало обзора.
% \startrelatedwork
% Пример ссылок в рамках обзора: \cite{example-english, example-russian, unrestricted-jump-evco, doerr-doerr-lambda-lambda-self-adjustment-arxiv}.
% %% Так помечается конец обзора.
% \finishrelatedwork
% Вне обзора:~\cite{bellman}. За одно проверим переносы в обычном тексте.
