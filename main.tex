\documentclass[times]{itmo-student-thesis}

%% Опции пакета:
%% - specification - если есть, генерируется задание, иначе не генерируется
%% - annotation - если есть, генерируется аннотация, иначе не генерируется
%% - times - делает все шрифтом Times New Roman, собирается с помощью xelatex
%% - languages={...} - устанавливает перечень используемых языков. По умолчанию это {english,russian}.
%%                     Последний из языков определяет текст основного документа.

%% Делает запятую в формулах более интеллектуальной, например:
%% $1,5x$ будет читаться как полтора икса, а не один запятая пять иксов.
%% Однако если написать $1, 5x$, то все будет как прежде.
\usepackage{icomma}

%% Один из пакетов, позволяющий делать таблицы на всю ширину текста.
\usepackage{tabularx}

%% Пакет для вставки pdf файлов в ваш теховский документ
\usepackage[final]{pdfpages} 

%% Начало
\usepackage{physics}
\usepackage{mathtools}
\usepackage{xcolor}
\usepackage{tcolorbox}
\usepackage{tabulary}
\usepackage{multirow}


%% Указываем файл с библиографией.
\addbibresource{bachelor-thesis.bib}

\begin{document}

\publishyear{2025}

% \studygroup{Z34434}
% \title{Оптомеханические эффекты при направленном возбуждении поверхностных плазмонов-поляритонов}
% \author{Бородулин Семен Сергеевич}{Бородулин С.С.}
% \supervisor{Костина Наталия Алексеевна}{Костина Н.А.}{канд. физ.-мат. наук}{младший научный сотрудник Университета ИТМО}
% \publishyear{2025}
% %% Дата выдачи задания. Можно не указывать, тогда надо будет заполнить от руки.
% \startdate{15}{января}{2025}
% %% Срок сдачи студентом работы. Можно не указывать, тогда надо будет заполнить от руки.
% \finishdate{25}{мая}{2025}
% %% Дата защиты. Можно не указывать, тогда надо будет заполнить от руки.
% \defencedate{15}{июня}{2019}

% \addconsultant{Петров М.И.}{канд. физ.-мат. наук, доцент Университета ИТМО}

%% Эта команда генерирует титульный лист и аннотацию (Нет, я её оставил, чтобы не пришлось переписывать заново всё и просто убрал вывод титульника, задания и аннотации).
\maketitle{Бакалавр}

%%Уберите знак % у трёх нижних строчек и загрузите ваши 3 pdf файла сюда и поменяйте название - тогда у вас будут ваши титульники, задания и аннотации из ИСУ
%\includepdf[pages=-]{Титульник.pdf}
%\includepdf[pages=-]{Задание ВКР.pdf}
%\includepdf[pages=-]{Аннотация.pdf}

%% Оглавление
\tableofcontents

\specialchapter{СПИСОК СОКРАЩЕНИЙ И УСЛОВНЫХ ОБОЗНАЧЕНИЙ\label{chapter:notation}}
\setlength{\parindent}{0cm} %убираем по ГОСТУ абзацный отступ
ППП --- поверхностный плазмон-поляритон;

ПВ --- поверхностная волна;

ЭМ --- Электромагнитный;

НИР --- Научно-исследовательская работа.

\setlength{\parindent}{1.25cm} %возвращаем абзацный отступ для остального доумента
%% Макрос для введения. Совместим со старым стилевиком.

\startprefacepage

%Современные исследования в области квантовой оптики и науках о жизни, к примеру, биологические исследования живых клеток, а также создание наноструктур в приложениях электроники требуют точных и неинвазивных методов контроля над положением объектов нанометровых размеров. Оптический пинцет является наиболее простым и почти бесконтактным методом механического манипулирования такими объектами. Однако, для точного контроля положения объекта обычно требуются сильно сфокусированные пучки с высокой интенсивностью, которые могут повредить или разрушить структуру объекта. В данной научно-исследовательской работе предлагается новый метод оптического манипулирования нанообъектами с мультипольным резонансным откликом, основанный на возбуждении направленных поверхностных плазмон-поляритонов на границе раздела диэлектрик-металл. 

В настоящее время технологический прогресс требует уменьшения размера систем диагностики и биохимического анализа, в том числе для развития персонализированной медицины, микрофлюидных и фотонных устройств Melzer2020Jun. В этом контексте особое значение приобретают компактные и многофункциональные системы «на чипе». Оптическое управление является одним из наименее инвазивных и достаточно простых способов механического управления объектами, что показало свою эффективность, в том числе, в чиповых системах Li19. 

В данной работе предлагается новый метод оптического контроля наноразмерных объектов за счет возбуждения направленных поверхностных волн на границе раздела диэлектрик-металл. Создание этого метода обусловлено тем, что стандартные способы оптического удержания не всегда подходят для ми-частиц из-за их сложного оптического отклика. Одно из главных преимуществ данного метода заключается в реализации сильно локализованного поля за счет интерференции направленного \textit{ППП} (поверхностного плазмон-поляритона) и возбуждаемых мультипольных моментов самой частицы, что, в свою очередь, позволяет реализовать высокоточный захват, а также увеличить вклад градиентной составляющей оптической силы и её доминирования над силой рассеяния для широкого спектра частиц.

В других работах уже рассматривалось использование плазмон-поляритонов для усиления взаимодействия и локализации поля IntroCit1, IntroCit2. В этой работе рассматривается реактивный эффект Petrov2016Jan - возникновение оптической силы, действующей в направлении обратном распространению возбуждаемого ППП. Стоит заметить, что важно учитывать соотношение между градиентной составляющей силы, которая возникает из-за поверхностной волны, и силой рассеяния, которая всегда направлена вдоль плоской волны, падающей на рассеиватель. Поэтому одной из задач работы является исследование параметров системы, при которых градиентная составляющая силы будет подавлять часть, отвечающую за рассеяние плоской волны. Именно в этом случае результирующая сила, действующая на частицу, будет определяться направлением поверхностного плазмон-поляритона. 

В работах Sinev2018Oct,Sinev2020Mar,Afinogenov2020Aug,Asilevi2024 было показано, что модификация мультипольного отклика частицы, расположенной на границе раздела сред, позволяет получить аналог эффекта Керкера в плоскости раздела и направленную генерацию поверхностной волны с помощью ближних полей частицы. Эти явления связаны с тем, что частицы с различным мультипольным составом возбуждают поверхностные волны в различных направлениях. Вследствие этого градиентная часть силы, действующей на эти частицы, будет направлена в разные стороны, что позволяет реализовать рассматриваемый метод сортировки. Именно поэтому важно учитывать взаимное расположение мультипольных резонансов в спектре рассеяния наночастицы и резонанса плазмона-поляритона, возбуждаемого ближними полями частицы. Таким образом, возможна реализация перемещения различных (по размеру и материалу) наночастиц в диапазоне углов от 0 до $\pi$ в плоскости раздела сред, путем изменения направлений поверхностной волны и реактивной силы за счет вариации параметров системы.

В качестве рассеивателя выбрана кремниевая наночастица, расположенная над металлом и поддерживающая электрический и магнитный дипольные резонансы в рассматриваемом диапазоне длин волн $\lambda=450-700$ нм. Изменение параметров излучения, таких как длина волны и поляризация падающего излучения, приводит к изменению амплитуды рассеяния от электрического и магнитного диполей, что, в свою очередь, меняет диаграмму направленности рассеянного излучения. Это также влияет на направление возбуждения поверхностных плазмон-поляритонов и, следовательно, на направление действующей оптической силы.


%% Начало содержательной части.

%%Костыль, копировать этот блок для создания новой ГЛАВЫ, она же РАЗДЕЛ (для обычных подразделов - пункты и подпункты - можно не копировать)
% \begin{nohyphens}
% {
% \titleformat{\chapter}[block]{\normalsize\bfseries}{\thechapter~}%
% {0pt}{}[\thispagestyle{\TheOnlyTruePageStyle}]
% \titlespacing{\chapter}{1.25cm}{0pt}{0pt}

\chapter{Введение}

% \titleformat{\chapter}[block]{\normalsize\bfseries\center}{\thechapter~}%
% {0pt}{}[\thispagestyle{\TheOnlyTruePageStyle}]
% \titlespacing{\chapter}{0pt}{-30pt}{0pt}
% }
% \end{nohyphens}

%% Так помечается начало обзора.
% \startrelatedwork
% Пример ссылок в рамках обзора: \cite{example-english, example-russian, unrestricted-jump-evco, doerr-doerr-lambda-lambda-self-adjustment-arxiv}.
% %% Так помечается конец обзора.
% \finishrelatedwork
% Вне обзора:~\cite{bellman}. За одно проверим переносы в обычном тексте.


\chapter{Методы}


\section{Теория}\label{sec:theory}

\subsection{Рассеяние от резонансных частиц}\label{subsec:scat_cs}
При помещении частицы в электромагнитное поле падающее излучение возбуждает в частице вынужденные колебания электрических зарядов (\textit{токи}), которые становятся источниками вторичных волн. Таким образом излучение изменяет свое направление и угловое распределение интенсивности при взаимодействии с нано-размерными объектами, иными словами происходит рассеяние излучения. При этом важна интерференция между падающей и вторичной волной. В работе рассматривается рассеяние монохроматических плоских волн на сферической нанометровой частице. Размеры рассеивателей сопоставимы с длиной падающей волны ($R \sim \lambda$), поэтому рассматривается теория рассеяния Ми (референс). 

Ми-частицы обладают сложной нелинейной зависимостью углового распределения интенсивности от длины волны (референс) и имеют резонансный характер рассеяния.

В качестве количественной характеристика рассеяния обычно рассматривают \textit{сечение рассеяния}. В качестве одного из определений сечения рассеяния можно рассмотреть отношения мощности рассеянного излучения к интенсивности падающего. Таким образом, сечение рассеяния $\sigma_{\text{sc}}$ определяется как
\begin{equation}
    \sigma_{\text{sc}} = \frac{\vb{P_{\text{sc}}}}{I_0} = \frac{1}{I_0} \oint_A  \left(  \mathbf{S}_{\text{sc}} \cdot \vb{n} \right) da,
\end{equation}

где $I_0$ - интенсивность падающего излучения, $P_{\text{sc}}$ - мощность рассеянного излучения, $\vb{S}_{\text{sc}}$ - вектор Пойнтинга рассеянного излучения, $S$ - поверхность, содержащая рассеиватель.

В данной работе падающее излучение - плоская волна, её интенсивность определяется по формуле
\begin{equation}
    I_0 = \frac{c n \varepsilon_0}{2} \left| \vb{E}_0 \right|^2,
\end{equation}

где $n$ - показатель преломления среды, $c$ - скорость света в вакууме,$\left| \vb{E}_0 \right|$ - амплитуда электрического поля падающей волны.
В работе рассматриваются кремниевые частицы. Мнимая часть коэффициента поглощения кремния очень мала $\Im{n_{\text{Si}}} \approx 0$ (сслыку на справочник), поэтому мощностью поглощенного излучения можно пренебречь. 

\subsection{Мультипольное разложение}\label{subsec:mult_decomp}

Мультипольное разложения является эффективном инстурментом для анализа электромагнитного поля в теории рассеяния. Оно позволяет разделять поле на компоненты и классифицировать их, что упрощает решение задач, например, связанных с взаимодействием электромагнитных волн с объектами. Суть метода заключается в разложение рассматриваемого поля в ряд сферических гармоник (ссылка на  рукштуля с точным разложением). Такое разложение позволяет выделить компоненты вносящие главный вклад в общее распределение поля и упростить решение поставленной задачи.
Рассматриваемые частицы имеют нанометровый размер и являются резонансными в диапазоне длин волн $\lambda = 300-1200$ нм. Их резонансное поведение хорошо описывается с использованием мультипольного разложения. 

Таким образом рассеянное наночастицей поле можно представить в виде следующего ряда:
\begin{equation}
    \vb{E}_{\text{sc}} (\vb{r}) = Z \sum_{j=1}^{\infty} \sum_{m=-j}^{j} i \hat{a}_{jm} \vb{N}_{jm}^{(3)}(\vb{r}) + \hat{b}_{jm} \vb{M}_{jm}^{(3)} (\vb{r}),\label{eq:multipolar_field}
\end{equation}
где $\hat{a}_{jm},\;\hat{b}_{jm}$ - коэффициенты характеризующие вклад от электрических $\vb{N}_{jm}^{(3)}$ и магнитных $\vb{M}_{jm}^{(3)}$ векторных сферических гармоник. 

Запишем общее выражение подсчета мультипольных коэффициентов, используя компоненты плотности тока:

\begin{align}
    \frac{(2\pi)^2}{4\pi} a_{jm} = \sum_{ln} (-i)^l \int d \hat{\vb{p}} \vb{Z}^{\dag}_{jm}(\hat{\vb{p}}) Y_{ln}(\hat{\vb{p}}) \int d^3 \vb{r} \vb{J}(\vb{r}) Y_{ln}^* (\hat{\vb{r}}) j_l(kr),\\
    \frac{(2\pi)^2}{4\pi} b_{jm} = \sum_{ln} (-i)^l \int d \hat{\vb{p}} \vb{X}^{\dag}_{jm}(\hat{\vb{p}}) Y_{ln}(\hat{\vb{p}}) \int d^3 \vb{r} \vb{J}(\vb{r}) Y_{ln}^* (\hat{\vb{r}}) j_l(kr),
\end{align}

где $\vb{J}(\vb{r})$ - плотность индуцированного тока, $Y_{l,m}(\hat{\vb{p}})$ - сферическая функция, $j_l(kr)$ - функция Бесселя 1-го рода порядка $l$, $\vb{X}_{jm}(\hat{\vb{p}})$ и $\vb{Z}_{jm}(\hat{\vb{p}})$ - векторные мультипольные функции:
\begin{equation}
    \begin{aligned}
        \vb{X}_{jm}(\hat{\vb{p}}) = \frac{1}{\sqrt{j(j+1)}} \vb{L} Y_{lm}(\hat{\vb{p}}), \\
        \vb{Z}_{jm}(\hat{\vb{p}}) = i \hat{\vb{p}} \times \vb{X}_{jm}(\vb{p}),
    \end{aligned}
\end{equation}
где компоненты вектора $\vb{L}$ - операторы углового момента, $\hat{\vb{p}}$ - нормированный вектор импульса $(|\vb{p}| = \omega/c)$.

Используя такое разложение поля, сечение рассеяния можно записать в следующей форме:
\begin{equation}
    \sigma_{\text{sc}} = \frac{1}{k^2} \sum_{j=1}^{\infty} \sum_{m=-j}^j |\hat{a}_{jm} |^2 + |\hat{b}_{jm}|^2 \label{eq:scat_sc}
\end{equation}

В формуле \eqref{eq:multipolar_field} содержатся коэффициенты мультипольного разложения для электромагнитного поля $\{\hat{a}_{jm}, \hat{b}_{jm}\}$. Однако, обычно используют мультипольные коэффициенты разложения плотности индуцированного тока $\{a_{jm}, b_{jm}\}$. Эти коэффициенты связаны с коэффициентами для поля следующим соотношением:
\begin{align}
    \hat{a}_{jm} &= -(i)^{j-1} \frac{k^2}{2 \sqrt{2 \pi}} a_{jm},\\
    \hat{b}_{jm} &= -(i)^j \frac{k^2}{2 \sqrt{2 \pi}} b_{jm},
\end{align}
где $k$ - волновое число падающего излучения. 

Используя коэффициенты разложения $\{a_{jm}, b_{jm}\}$, можно записать выражения для индуцированного электрического дипольного момента в декартовой системе координат (ссылка на рукштуля с приближенными моментами):

\begin{equation}
    \begin{bmatrix}
        p_x \\
        p_y \\
        p_z
    \end{bmatrix} = C^e_1 \begin{bmatrix}
        \frac{a_{1-1} - a_{11}}{\sqrt{2}},\\
        \frac{a_{1-1}+a_{11}}{\sqrt{2}i},\\
        a_{10},
    \end{bmatrix}
\end{equation}

где $C_1^e = \frac{\sqrt{3} \pi}{i \omega}$, $\omega$ - частота падающего излучения.

Аналогичные выражения можно записать для магнитного дипольного момента:
\begin{equation}
    \begin{bmatrix}
        m_x \\
        m_y \\
        m_z
    \end{bmatrix} = C_1^m \begin{bmatrix}
        \frac{b_{1-1} - b_{11}}{\sqrt{2}},\\
        \frac{b_{1-1}+b_{11}}{\sqrt{2}i},\\
        b_{10},
    \end{bmatrix}
\end{equation}
где $C_1^m = -\frac{\sqrt{3} \pi}{k}$.

Рассмотрим выражения для мультипольных моментов следующего порядка - квадруполей:

\begin{align*}
    Q_{xx}^m &= C^m_2 \left[ \frac{b_{22}+b_{2-2}}{2} - \frac{b_{20}}{\sqrt{6}} \right], \\
    Q_{xy}^m = Q_{yx}^m &= C^m_2 \left[ \frac{b_{2-2}-b_{22}}{2 i} \right], \\
    Q_{xz}^m = Q_{zx}^m &= C^m_2 \left[ \frac{b_{2-1}-b_{21}}{2}  \right],\\
    Q_{yz}^m = Q_{zy}^m &= C^m_2 \left[ \frac{b_{2-1}+b_{21}}{2i} \right],\\
    Q_{yy}^m &= C^m_2 \left[ \left( \frac{b_{22}+b_{2-2}}{2} \right)  -\frac{b_{20}}{\sqrt{6}}\right],\\
    Q_zz^m &= C^m_2 \frac{2 }{\sqrt{6}} b_{20} = -Q_{xx}^m -Q_{yy}^m,
\end{align*}
где $C_2^m = \frac{6 \pi \sqrt{10}}{i k^2}$. 

\begin{align*}
    Q_{xx}^e &= C_2^e \left[ \frac{a_{22}+a_{2-2}}{2} - \frac{a_{20}}{\sqrt{6}} \right], \\
    Q_{xy}^e = Q_{yx}^e &= C_2^e \left[ \frac{a_{2-2}-a_{22}}{2 i} \right], \\
    Q_{xz}^e = Q_{zx}^e &= C_2^e \left[ \frac{a_{2-1}-a_{21}}{2}  \right],\\
    Q_{yz}^e = Q_{zy}^e &= C_2^e \left[ \frac{a_{2-1}+a_{21}}{2i} \right],\\
    Q_{yy}^e &= C_2^e \left[ \left( \frac{a_{22}+b_{2-2}}{2} \right)  -\frac{a_{20}}{\sqrt{6}}\right],\\
    Q_zz^e &= C_2^e \frac{2 }{\sqrt{6}} a_{20} = -Q_{xx}^e -Q_{yy}^e,
\end{align*}
где $C_2^e = \frac{6 \pi \sqrt{10}}{c k^2}$, $c$ - скорость света в вакууме. 

Используя полученные мультипольные моменты можно модифицировать выражение \eqref{eq:scat_sc}:
\begin{align*}
    \sigma_{\text{sc}} &= \sigma_{\text{sc}}^p + \sigma_{\text{sc}}^m + \sigma_{\text{sc}}^{Q^e} +
    \sigma_{\text{sc}}^{Q^m} + ... = \\ &= \frac{k^4}{6 \pi \varepsilon_0^2 |\vb{E}_0|^2 } \left[ \sum_{\alpha}\left( |p_{\alpha}|^2 + 
    \frac{| m_{\alpha}|^2}{c^2} \right) + \frac{1}{120} \sum_{\alpha,\beta} \left( |k Q_{\alpha \beta}^e|^2 + |\frac{k Q_{\alpha \beta}^m}{c}|^2 \right) + ...\right]
\end{align*}

Эта формула сечения рассеяния позволяет разделить вклад в спектр от различных мультипольных моментов. 

\subsection{Поверхностный плазмон-поляритон}\label{subsec:spp}

Металлические материалы обладают отрицательной диэлектрической проницаемостью на оптических частотах, что приводит, например, к высокому отражению света. Кроме этого, в оптическом диапазоне частот электронный газ в металле поддерживает колебания поверхностной плотности заряда при резонансных частотах, что называется \textit {поверхностным плазмон-поляритоном} (ссылка на основы нанооптики). Термин плазмон-поляритон характеризует взаимодействие фотонов с коллективными возбуждениями электронов в металле. Осцилляция плотности поверхностного заряда, появляющиеся вследствие поверхностных плазмон-поляритонов на границе  диэлектрик-металл, могут привести к сильному усилению локального поля вблизи границы. 

(Сюда схемку/картинку)

Рассмотри границу раздела диэлектрика $\varepsilon_1$ и металла $\varepsilon_2$. Расположим границу раздела при $z=0$ в декартовой системе координат. Найдем однородные решения уравнений Максвелла, которые локализованы на границе раздела.  Иными сломами найдем собственные моды системы, которые могут распространяться вдоль границы раздела и затухают по мере отдаления от неё. Волновое уравнение:
\begin{equation}
    \nabla^2 \vb{E}(\vb{r}, \omega) - \frac{\omega^2}{c^2} \varepsilon(\vb{r}, \omega) \vb{E}(\vb{r}, \omega) = 0, \label{eq;wave_eq_spp}
\end{equation}

где $\varepsilon(\vb{r}, \omega) = \varepsilon_1 (\omega)$ если $z<0$ и $\varepsilon(\vb{r}, \omega) = \varepsilon_2 (\omega) $ при $z<0$. 
Решение уравнения - p-поляризованные волны:

\begin{equation}
    \vb{E}_j = \begin{pmatrix}
        E_{j,x} \\
        0 \\
        E_{j,z}
    \end{pmatrix} e^{i(k_x x - \omega t)} e^{i k_{j,z} z},
\end{equation}
где $j=1,2$ - характеризует область диэлектрика и металла. Так как волновой вектор имеет только $x$ и $z$ компоненту, тогда связь между компонентами выглядит следующим образом: 
\begin{equation}
    k_x^2 + k_{j,z}^2 = \frac{\omega^2}{c^2} \varepsilon_j, j=1,2. \label{eq:wavevect_spp}
\end{equation}
Условие отсутствия источников $\nabla \cdot \vb{E} = 0$ приводит к следующему выражению:
\begin{equation}
    k_x E_{j,x} + k_{j,z} E_{i,z} = 0, j=1,2. \label{eq:source_free_spp}
\end{equation}
Используя условия непрерывности параллельной компоненты $\vb{E}$ и нормальной компоненты $\vb{D}$ на границе раздела, получаем:
\begin{equation}
    \begin{aligned}
    E_{1,x} - E_{2,x} &= 0,\\
    \varepsilon_1 E_{1,z} - \varepsilon_2 E_{2,z} &= 0. \label{eq:boundary_conditions_spp}
    \end{aligned}
\end{equation}
Совмещая выражения \eqref{eq:source_free_spp} и \eqref{eq:boundary_conditions_spp}, получим систему из четырёх линейных уравнений, которая имеет нетривиальное решение при условии:
\begin{equation}
    \varepsilon_1 k_{2,z} + \varepsilon_2 k_{1,z} = 0. \label{eq:spp_condition}
\end{equation}
Из \eqref{eq:spp_condition} и \eqref{eq:wavevect_spp} получим дисперсионное соотношение для поверхностного плазмон-поляритона:
\begin{equation}
    \begin{aligned}
        k_x^2 &= \frac{\varepsilon_1 \varepsilon_2}{\varepsilon_1 + \varepsilon_2} \frac{\omega^2}{c^2}, \label{eq:spp_dispersion} \\
        k_{j,z}^2 &= \frac{\varepsilon_j^2}{\varepsilon_1+\varepsilon_2} k^2, j=1,2.
    \end{aligned}
\end{equation}

Заметим, что $\Re{k_x}$ - отвечает за длину волны возбуждаемого плазмона, в то время как $\Im{k_x}$ характеризует степень затухания плазмона по мере распространения вдоль оси $x$.
(график теор. зависимости k(w) нужно?)

Получим условия существования поверхностного плазмона. Для простоты пренебрежем мнимыми частями диэлектрических проницаемостей.  Мы ищем распространяющиеся вдоль оси $x$ моды, поэтому требуется, чтобы $k_x^2$ было вещественным. Это можно получить при условии, если произведение и сумма диэлектрических проницаемостей имеют один и тот же знак \eqref{eq:spp_dispersion}. Так как ищем решение, которое экспоненциально затухает с увеличением расстояния от границы раздела - $k_{j,z}$ должно быть мнимым числом. Из этого получаются следующие условия возбуждения поверхностного плазмон-поляритона:

\begin{equation}
    \begin{aligned}
        \varepsilon_1(\omega) \cdot \varepsilon_2(\omega) < 0,\\
        \varepsilon_1(\omega) + \varepsilon_2(\omega) <0
    \end{aligned}
\end{equation}

Таким образом, условие существования поверхностного плазмон-поляритона - наличие материала, который имеет отрицательную вещественную часть диэлектрической проницаемости, которая превосходит по модулю величину диэлектрической проницаемости другого материала. Такими свойствами обладает золото или серебро (привести график дисперсионки). Все материалы немагнитные, поэтому $\mu=1$.

\subsection{Полное поле}\label{subsec:full_field}

Рассмотрим систему из границы раздела и частицы, расположенной в верхнем полупространстве. (Схему добавить) Получим аналитические выражения дипольных моментов с помощью функции Грина для отраженного поля. 

Полное поле, действующее на частицу, можно записать в следующем виде:
\begin{equation}
    \begin{aligned}
        \vb{E}_{tot} = \vb{E}^0_{inc} + \vb{E}^0_{ref} + \vb{E}^{\vb{p}}_{ref}+\vb{E}^{\vb{m}}_{ref},\\
        \vb{H}_{tot} = \vb{H}^0_{inc} + \vb{H}^0_{ref} + \vb{H}^{\vb{p}}_{ref}+\vb{H}^{\vb{m}}_{ref},
    \end{aligned}
\end{equation}
где $\vb{E}^0_{inc},\;\vb{H}^0_{inc}$ - падающее электрическое и магнитное поле, $\vb{H}^0_{inc},\; \vb{H}^0_{ref}$ - отраженное электрическое и магнитное поле, $\vb{E}^{\vb{p},\vb{m}}_{ref},\;\vb{H}^{\vb{p},\vb{m}}_{ref}$ - отраженные электрические и магнитные поля от соответсвующих наведенных дипольных моментов, которые можно выразить через функцию Грина для отраженного излучения:

\begin{align}
    \vb{E}_{ref}^{\vb{p}} &= \frac{ k^2}{\varepsilon_0} \hat{\vb{G}}_{ref}^E (\vb{r}_0, \vb{r}_0) \vb{p} = \hat{\vb{G}}^{ref}_{ee} (\vb{r}_0, \vb{r}_0) \vb{p},\\
    \vb{E}_{ref}^{\vb{m}} &= i\omega  \mu_0 [\nabla \times \hat{\vb{G}}_{ref}^H (\vb{r}_0, \vb{r}_0) ]\vb{m} = \hat{\vb{G}}^{ref}_{em} (\vb{r}_0, \vb{r}_0)\vb{m},\\
    \vb{H}_{ref}^{\vb{m}} &=\varepsilon  k^2 \hat{\vb{G}}_{ref}^H (\vb{r}_0, \vb{r}_0) \vb{m} = \hat{\vb{G}}^{ref}_{mm}(\vb{r}_0, \vb{r}_0) \vb{m},\\
    \vb{H}_{ref}^{\vb{p}} &= - i\omega [\nabla \times \hat{\vb{G}}_{ref}^E (\vb{r}_0, \vb{r}_0) ]\vb{p} =  \hat{\vb{G}}^{ref}_{me}(\vb{r}_0, \vb{r}_0) \vb{p},
\end{align}
где $\vb{r}_0$ - точка положения центра частицы.

Так как мы рассматриваем поля в верхнем полупространстве, которое соответствует воздуху, дальше примем $\varepsilon_1=1$.

Диадная функция Грина для нашей задачи выглядит следующим образом:


\begin{equation}
    \hat{\vb{G}}_{ref}^{E,H} = \frac{i}{8 \pi^2} \iint ^{\infty}_{-\infty} [ \hat{\vb{M}}^{s}_{ref} + \hat{\vb{M}}_{ref}^{p}] e^{i(k_x (x-x_0) + k_y(y-y_0) + k_z (z+z_0))} dk_x dk_y \label{eq:green_func}
\end{equation}

Матрица S-поляризованного излучения:
\begin{equation}
    \hat{\vb{M}}_{ref}^{s} = \frac{r^s_{E,H}}{k_z (k_x^2+k_y^2)}\begin{bmatrix}
        k_y^2 & -k_x k_y & 0 \\
        -k_x k_y & k_x^2 & 0 \\
         0 & 0 & 0\\
    \end{bmatrix}\label{eq:s-pol-m-green}
\end{equation}

Матрица P-поляризованного излучения. 
\begin{equation}
    \hat{\vb{M}}_{ref}^{p} = -\frac{r^p_{E,H}}{k^2 (k_x^2+k_y^2)}\begin{bmatrix}
        k_x^2 k_z & k_x k_y k_z & k_x (k_x^2+k_y^2) \\
        k_x k_y k_z & k_y^2 k_z & k_y (k_x^2+k_y^2) \\
        -k_x (k_x^2+k_y^2) & -k_y (k_x^2+k_y^2) & -(k_x^2+k_y^2)^2/k_z\\
    \end{bmatrix}\label{eq:p-pol-m-green}
\end{equation}

В выражениях \eqref{eq:s-pol-m-green} и \eqref{eq:p-pol-m-green} $k_x$, $k_y$, $k_z$ - компоненты волнового вектора в декартовой системе координат, а $r^s$, $r^p$ - амплитудные коэффициенты отражения Френеля:
\begin{equation}
    r_E^s(k_\rho) = r^p_H (k_\rho) = \frac{k_{z,1} - k_{z,2}}{k_{z,1}+k_{z,2}} \; r_E^p(k_\rho) = r_H^s (k_\rho) = \frac{\varepsilon_2 k_{z,1} - \varepsilon_1 k_{z,2}}{\varepsilon_2 k_{z,1} + \varepsilon_1 k_{z,2}}, \label{eq:frenel_coeff}
\end{equation} 
где $k_{z,j} = \sqrt{k^2_j - k_\rho^2} = \sqrt{\varepsilon_j k_0 - k_\rho^2}$, $k_\rho$ - продольная составляющая волнового вектора, $j=1,2$ - соответствуют верхнему и нижнему полупространству. Преобразуем и упростим выражения функций Грина \eqref{eq:green_func}, перейдя в цилиндрическую систему координат:

\begin{equation}
    \hat{\vb{G}}_{ref}^E (\vb{r}_0, \vb{r}_0) = \frac{i}{ 8 \pi k^2}\int_0^{\infty} \frac{k_\rho}{k_z} \begin{bmatrix}
        k^2 r^s - k_z^2 r^p & 0 & 0\\
        0 & k^2 r^s - k_z^2 r^p & 0 \\
        0& 0 & 2 k_\rho^2 r^p \\
    \end{bmatrix}
    e^{2ik_z z_0} dk_\rho,
\end{equation}

\begin{equation}
    \hat{\vb{G}}_{ref}^H (\vb{r}_0, \vb{r}_0) = \frac{i}{8 \pi k^2}\int_0^{\infty} \frac{k_\rho}{k_z} \begin{bmatrix}
        k^2 r^p - k_z^2 r^s & 0 & 0\\
        0 & k^2 r^p - k_z^2 r^s & 0 \\
        0& 0 & 2 k_\rho^2 r^s \\
    \end{bmatrix}
    e^{2ik_z z_0} dk_\rho.
\end{equation}

Используя эти выражения можно получить полное поле, действующее на частицу и получить выражения для дипольных моментов:

\begin{align}
    \vb{p} &= \varepsilon_0 \alpha_E \vb{E}_{tot} = \varepsilon_0 \alpha_E \left(\vb{E}^0 + \hat{\vb{G}}^{ref}_{ee} ( \vb{r}_0, \vb{r}_0) \vb{p} +  \hat{\vb{G}}^{ref}_{em} (\vb{r}_0, \vb{r}_0) \vb{m} \right),\\
    \vb{m} &= \alpha_H \vb{H}_{tot} = \alpha_H \left(\vb{H}^0   + \hat{\vb{G}}^{ref}_{mm} (\vb{r}_0, \vb{r}_0) \vb{m} -  \hat{\vb{G}}^{ref}_{me} (\vb{r}_0, \vb{r}_0) \vb{p}  \right),\label{eq:dipoles-full-field} 
\end{align}
где $\varepsilon_0$ - диэлектрическая постоянная, $\alpha_{E,H}$ - эффективные поляризуемости для сферы, выражаемые через коэффициенты рассеяния Ми $a_1,\;b_1$. $\alpha_E = 6i\pi a_1/k^3$, $\alpha_H = 6i\pi b_1/k^3$.

С другой стороны, дипольные моменты можно выразить через начальное падающее и отраженные поля и тензоры эффективной поляризуемости (по аналогии с бианизотропией):

\begin{equation}
    \begin{aligned}
        \vb{p} &= \varepsilon_0 \hat{\alpha}_{ee} \vb{E}^0 + \hat{\alpha}_{em} \vb{H}^0,  \\
        \vb{m} &=   \varepsilon_0 \hat{\alpha}_{me} \vb{E}^0+ \hat{\alpha}_{mm} \vb{H}^0_{inc}. \label{eq:dipoles-tensors} 
    \end{aligned}
\end{equation}

Получили самосогласованную задачу/ Подставим в выражения \eqref{eq:dipoles-full-field} дипольные моменты из \eqref{eq:dipoles-tensors} и получим формулы для тензоров эффективной поляризуемости:

\begin{equation}
    \begin{aligned}
        \hat{\alpha}_{ee} = \hat{\vb{A}}^{-1} \alpha_E, \;\;
        \hat{\alpha}_{em} = \hat{\vb{A}}^{-1} \varepsilon_0 \alpha_E \hat{\vb{G}}_{em}^{ref} ( \hat{\vb{I}} - \alpha_H \hat{\vb{G}}_{mm}^{ref})^{-1} \alpha_H,\\
        \hat{\alpha}_{me} = \hat{\vb{B}}^{-1} \alpha_H \hat{\vb{G}}^{ref}_{me} (\hat{\vb{I}} - \varepsilon_0 \alpha_E \hat{\vb{G}}_{ee}^{ref} )^{-1} \varepsilon_0 \alpha_E, \;\;
        \hat{\alpha}_{ee} = \hat{\vb{B}}^{-1} \alpha_H,
    \end{aligned}\label{eq:polaris-tensors}
\end{equation}

где $\hat{\vb{I}}$ единичный тензор, $\hat{\vb{A}},\; \hat{\vb{B}}$ - записываются как:
\begin{equation}
    \begin{aligned}
        \hat{\vb{A}} &= \hat{\vb{I}} - \varepsilon_0 \alpha_E \hat{\vb{G}}_{ee}^{ref} - \varepsilon_0 \alpha_E \hat{\vb{G}}_{em}^{ref} ( \hat{\vb{I}} - \alpha_H \hat{\vb{G}}_{mm}^{ref})^{-1} \alpha_H \hat{\vb{G}}_{me}^{ref},\\
        \hat{\vb{B}} &= \hat{\vb{I}} - \alpha_H \hat{\vb{G}}_{mm}^{ref} - \alpha_H \hat{\vb{G}}^{ref}_{me} (\hat{\vb{I}} - \varepsilon_0 \alpha_E \hat{\vb{G}}_{ee}^{ref} )^{-1} \varepsilon_0 \alpha_E \hat{\vb{G}}_{em}^{ref}.
    \end{aligned}
\end{equation}




% \section{Начальное поле}\label{sec:inc_field}

% \section{Сечение рассеяния и экстинкции}


% \section{Мультипольное разложение}

% \section{Оптическая сила}

% \section{Функции Грина}

% \section{Поверхностный плазмон}

% \chapterconclusion


\chapter{Результаты}


%% Макрос для заключения. Совместим со старым стилевиком.
\startconclusionpage

В данном разделе размещается заключение.

\printmainbibliography

%% После этой команды chapter будет генерировать приложения, нумерованные русскими буквами.
%% \startappendices из старого стилевика будет делать то же самое
\appendix

\chapter{Пример приложения}\label{sec:app:1}

\end{document}

