\chapter{Методы}


\section{Теория}\label{sec:theory}

\subsection{Рассеяние от резонансных частиц}\label{subsec:scat_cs}

В качестве количественной характеристика рассеяния обычно рассматривают \textit{сечение рассеяния}. В качестве одного из определений сечения рассеяния можно рассмотреть отношения мощности рассеянного излучения к интенсивности падающего. Таким образом, сечение рассеяния $\sigma_{\text{sc}}$ определяется как
\begin{equation}
    \sigma_{\text{sc}} = \frac{\vb{P_{\text{sc}}}}{I_0} = \frac{1}{I_0} \oint_A  \left(  \mathbf{S}_{\text{sc}} \cdot \vb{n} \right) da,
\end{equation}

где $I_0$ - интенсивность падающего излучения, $P_{\text{sc}}$ - мощность рассеянного излучения, $\vb{S}_{\text{sc}}$ - вектор Пойнтинга рассеянного излучения, $S$ - поверхность, содержащая рассеиватель.

В данной работе падающее излучение - плоская волна, её интенсивность определяется по формуле
\begin{equation}
    I_0 = \frac{c n \varepsilon_0}{2} \left| \vb{E}_0 \right|^2,
\end{equation}

где $n$ - показатель преломления среды, $c$ - скорость света в вакууме,$\left| \vb{E}_0 \right|$ - амплитуда электрического поля падающей волны.
В работе рассматриваются кремниевые частицы. Мнимая часть коэффициента поглощения кремния очень мала $\Im{n_{\text{Si}}} \approx 0$ (сслыку на справочник), поэтому мощностью поглощенного излучения можно пренебречь. 

\subsection{Мультипольное разложение}\label{subsec:mult_decomp}

Мультипольное разложения является эффективном инстурментом для анализа электромагнитного поля в теории рассеяния. Оно позволяет разделять поле на компоненты и классифицировать их, что упрощает решение задач, например, связанных с взаимодействием электромагнитных волн с объектами. Суть метода заключается в разложение рассматриваемого поля в ряд сферических гармоник (ссылка на  рукштуля с точным разложением). Такое разложение позволяет выделить компоненты вносящие главный вклад в общее распределение поля и упростить решение поставленной задачи.
Рассматриваемые частицы имеют нанометровый размер и являются резонансными в диапазоне длин волн $\lambda = 300-1200$ нм. Их резонансное поведение хорошо описывается с использованием мультипольного разложения. 

Таким образом рассеянное наночастицей поле можно представить в виде следующего ряда:
\begin{equation}
    \vb{E}_{\text{sc}} (\vb{r}) = Z \sum_{j=1}^{\infty} \sum_{m=-j}^{j} i \hat{a}_{jm} \vb{N}_{jm}^{(3)}(\vb{r}) + \hat{b}_{jm} \vb{M}_{jm}^{(3)} (\vb{r}),\label{eq:multipolar_field}
\end{equation}
где $\hat{a}_{jm},\;\hat{b}_{jm}$ - коэффициенты характеризующие вклад от электрических $\vb{N}_{jm}^{(3)}$ и магнитных $\vb{M}_{jm}^{(3)}$ векторных сферических гармоник. 

Запишем общее выражение подсчета мультипольных коэффициентов, используя компоненты плотности тока:

\begin{align}
    \frac{(2\pi)^2}{4\pi} a_{jm} = \sum_{ln} (-i)^l \int d \vb{\hat{p}} \vb{Z}^{\dag}_{jm}(\vb{\hat{p}}) Y_{ln}(\vb{\hat{p}} \iiint d^3 \vb{r} \vb{J}(\vb{r})) Y_{ln}^* (\vb{\hat{r}}) j_l(kr),\\
    \frac{(2\pi)^2}{4\pi} b_{jm} = \sum_{ln} (-i)^l \int d \vb{\hat{p}} \vb{X}^{\dag}_{jm}(\vb{\hat{p}}) Y_{ln}(\vb{\hat{p}} \iiint d^3 \vb{r} \vb{J}(\vb{r})) Y_{ln}^* (\vb{\hat{r}}) j_l(kr),
\end{align}

Используя такое разложение поля, сечение рассеяния можно записать в следующей форме:
\begin{equation}
    \sigma_{\text{sc}} = \frac{1}{k^2} \sum_{j=1}^{\infty} \sum_{m=-j}^j |\hat{a}_{jm} |^2 + |\hat{b}_{jm}|^2 \label{eq:scat_sc}
\end{equation}

В формуле \eqref{eq:multipolar_field} содержатся коэффициенты мультипольного разложения для электромагнитного поля $\{\hat{a}_{jm}, \hat{b}_{jm}\}$. Однако, обычно используют мультипольные коэффициенты разложения плотности индуцированного тока $\{a_{jm}, b_{jm}\}$. Эти коэффициенты связаны с коэффициентами для поля следующим соотношением:
\begin{align}
    \hat{a}_{jm} &= -(i)^{j-1} \frac{k^2}{2 \sqrt{2 \pi}} a_{jm},\\
    \hat{b}_{jm} &= -(i)^j \frac{k^2}{2 \sqrt{2 \pi}} b_{jm},
\end{align}
где $k$ - волновое число падающего излучения. 

Используя коэффициенты разложения $\{a_{jm}, b_{jm}\}$, можно записать выражения для индуцированного электрического дипольного момента в декартовой системе координат (ссылка на рукштуля с приближенными моментами):

\begin{equation}
    \begin{bmatrix}
        p_x \\
        p_y \\
        p_z
    \end{bmatrix} = C^e_1 \begin{bmatrix}
        \frac{a_{1-1} - a_{11}}{\sqrt{2}},\\
        \frac{a_{1-1}+a_{11}}{\sqrt{2}i},\\
        a_{10},
    \end{bmatrix}
\end{equation}

где $C_1^e = \frac{\sqrt{3} \pi}{i \omega}$, $\omega$ - частота падающего излучения.

Аналогичные выражения можно записать для магнитного дипольного момента:
\begin{equation}
    \begin{bmatrix}
        m_x \\
        m_y \\
        m_z
    \end{bmatrix} = C_1^m \begin{bmatrix}
        \frac{b_{1-1} - b_{11}}{\sqrt{2}},\\
        \frac{b_{1-1}+b_{11}}{\sqrt{2}i},\\
        b_{10},
    \end{bmatrix}
\end{equation}
где $C_1^m = -\frac{\sqrt{3} \pi}{k}$.

Рассмотрим выражения для мультипольных моментов следующего порядка - квадруполей:

\begin{align*}
    Q_{xx}^m &= C^m_2 \left[ \frac{b_{22}+b_{2-2}}{2} - \frac{b_{20}}{\sqrt{6}} \right], \\
    Q_{xy}^m = Q_{yx}^m &= C^m_2 \left[ \frac{b_{2-2}-b_{22}}{2 i} \right], \\
    Q_{xz}^m = Q_{zx}^m &= C^m_2 \left[ \frac{b_{2-1}-b_{21}}{2}  \right],\\
    Q_{yz}^m = Q_{zy}^m &= C^m_2 \left[ \frac{b_{2-1}+b_{21}}{2i} \right],\\
    Q_{yy}^m &= C^m_2 \left[ \left( \frac{b_{22}+b_{2-2}}{2} \right)  -\frac{b_{20}}{\sqrt{6}}\right],\\
    Q_zz^m &= C^m_2 \frac{2 }{\sqrt{6}} b_{20} = -Q_{xx}^m -Q_{yy}^m,
\end{align*}
где $C_2^m = \frac{6 \pi \sqrt{10}}{i k^2}$. 

\begin{align*}
    Q_{xx}^e &= C_2^e \left[ \frac{a_{22}+a_{2-2}}{2} - \frac{a_{20}}{\sqrt{6}} \right], \\
    Q_{xy}^e = Q_{yx}^e &= C_2^e \left[ \frac{a_{2-2}-a_{22}}{2 i} \right], \\
    Q_{xz}^e = Q_{zx}^e &= C_2^e \left[ \frac{a_{2-1}-a_{21}}{2}  \right],\\
    Q_{yz}^e = Q_{zy}^e &= C_2^e \left[ \frac{a_{2-1}+a_{21}}{2i} \right],\\
    Q_{yy}^e &= C_2^e \left[ \left( \frac{a_{22}+b_{2-2}}{2} \right)  -\frac{a_{20}}{\sqrt{6}}\right],\\
    Q_zz^e &= C_2^e \frac{2 }{\sqrt{6}} a_{20} = -Q_{xx}^e -Q_{yy}^e,
\end{align*}
где $C_2^e = \frac{6 \pi \sqrt{10}}{c k^2}$, $c$ - скорость света в вакууме. 

Используя полученные мультипольные моменты можно модифицировать выражение \eqref{eq:scat_sc}:
\begin{align*}
    \sigma_{\text{sc}} &= \sigma_{\text{sc}}^p + \sigma_{\text{sc}}^m + \sigma_{\text{sc}}^{Q^e} +
    \sigma_{\text{sc}}^{Q^m} + ... = \\ &= \frac{k^4}{6 \pi \varepsilon_0^2 |\vb{E}_0|^2 } \left[ \sum_{\alpha}\left( |p_{\alpha}|^2 + 
    \frac{| m_{\alpha}|^2}{c^2} \right) + \frac{1}{120} \sum_{\alpha,\beta} \left( |k Q_{\alpha \beta}^e|^2 + |\frac{k Q_{\alpha \beta}^m}{c}|^2 \right) + ...\right]
\end{align*}

Используя данную формулу сечения рассеяния можно разделить вклад в спектр от различных мультипольных моментов. 

\subsection{Поверхностный плазмон-поляритон}\label{subsec:spp}






% \section{Начальное поле}\label{sec:inc_field}

% \section{Сечение рассеяния и экстинкции}


% \section{Мультипольное разложение}

% \section{Оптическая сила}

% \section{Функции Грина}

% \section{Поверхностный плазмон}

% \chapterconclusion
