\startprefacepage

%Современные исследования в области квантовой оптики и науках о жизни, к примеру, биологические исследования живых клеток, а также создание наноструктур в приложениях электроники требуют точных и неинвазивных методов контроля над положением объектов нанометровых размеров. Оптический пинцет является наиболее простым и почти бесконтактным методом механического манипулирования такими объектами. Однако, для точного контроля положения объекта обычно требуются сильно сфокусированные пучки с высокой интенсивностью, которые могут повредить или разрушить структуру объекта. В данной научно-исследовательской работе предлагается новый метод оптического манипулирования нанообъектами с мультипольным резонансным откликом, основанный на возбуждении направленных поверхностных плазмон-поляритонов на границе раздела диэлектрик-металл. 

В настоящее время технологический прогресс требует уменьшения размера систем диагностики и биохимического анализа, в том числе для развития персонализированной медицины, микрофлюидных и фотонных устройств Melzer2020Jun. В этом контексте особое значение приобретают компактные и многофункциональные системы «на чипе». Оптическое управление является одним из наименее инвазивных и достаточно простых способов механического управления объектами, что показало свою эффективность, в том числе, в чиповых системах Li19. 

В данной работе предлагается новый метод оптического контроля наноразмерных объектов за счет возбуждения направленных поверхностных волн на границе раздела диэлектрик-металл. Создание этого метода обусловлено тем, что стандартные способы оптического удержания не всегда подходят для ми-частиц из-за их сложного оптического отклика. Одно из главных преимуществ данного метода заключается в реализации сильно локализованного поля за счет интерференции направленного \textit{ППП} (поверхностного плазмон-поляритона) и возбуждаемых мультипольных моментов самой частицы, что, в свою очередь, позволяет реализовать высокоточный захват, а также увеличить вклад градиентной составляющей оптической силы и её доминирования над силой рассеяния для широкого спектра частиц.

В других работах уже рассматривалось использование плазмон-поляритонов для усиления взаимодействия и локализации поля IntroCit1, IntroCit2. В этой работе рассматривается реактивный эффект Petrov2016Jan - возникновение оптической силы, действующей в направлении обратном распространению возбуждаемого ППП. Стоит заметить, что важно учитывать соотношение между градиентной составляющей силы, которая возникает из-за поверхностной волны, и силой рассеяния, которая всегда направлена вдоль плоской волны, падающей на рассеиватель. Поэтому одной из задач работы является исследование параметров системы, при которых градиентная составляющая силы будет подавлять часть, отвечающую за рассеяние плоской волны. Именно в этом случае результирующая сила, действующая на частицу, будет определяться направлением поверхностного плазмон-поляритона. 

В работах Sinev2018Oct,Sinev2020Mar,Afinogenov2020Aug,Asilevi2024 было показано, что модификация мультипольного отклика частицы, расположенной на границе раздела сред, позволяет получить аналог эффекта Керкера в плоскости раздела и направленную генерацию поверхностной волны с помощью ближних полей частицы. Эти явления связаны с тем, что частицы с различным мультипольным составом возбуждают поверхностные волны в различных направлениях. Вследствие этого градиентная часть силы, действующей на эти частицы, будет направлена в разные стороны, что позволяет реализовать рассматриваемый метод сортировки. Именно поэтому важно учитывать взаимное расположение мультипольных резонансов в спектре рассеяния наночастицы и резонанса плазмона-поляритона, возбуждаемого ближними полями частицы. Таким образом, возможна реализация перемещения различных (по размеру и материалу) наночастиц в диапазоне углов от 0 до $\pi$ в плоскости раздела сред, путем изменения направлений поверхностной волны и реактивной силы за счет вариации параметров системы.

В качестве рассеивателя выбрана кремниевая наночастица, расположенная над металлом и поддерживающая электрический и магнитный дипольные резонансы в рассматриваемом диапазоне длин волн $\lambda=450-700$ нм. Изменение параметров излучения, таких как длина волны и поляризация падающего излучения, приводит к изменению амплитуды рассеяния от электрического и магнитного диполей, что, в свою очередь, меняет диаграмму направленности рассеянного излучения. Это также влияет на направление возбуждения поверхностных плазмон-поляритонов и, следовательно, на направление действующей оптической силы.
